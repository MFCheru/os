\section{Read-Write Lock}

Un \textit{Read-Write Lock} es una primitiva de sincronización diseñada con el fin de proveer bajo paralelismo estabilidad a las estructuras de datos, motivada especialmente por los efectos de la concurrencia.
Su objetivo consiste en facilitar una serie de funciones que permitan coordinar efectivamente el uso compartido de ciertos datos, permitiendo paralelismo unicamente para la lectura, y otorgando un solo permiso de escritura; para ello cada \textit{thread} deberá invocar funciones de \textit{lock} y \textit{unlock} de lectura/escritura, que soliciten y liberen respectivamente el permiso de acceso.   
A diferencia de otros mecanismos como el \textit{Spin Lock}, el \textit{Read-Write Lock} evita las situaciones de \textit{polling}, permitiendo que los \textit{thread} que permanezcan esperando el permiso de acceso, tras invocar un \textit{lock} se mantengan dormidos hasta que el \textit{thread} correspondiente libere dicho permiso. Por este motivo suele implementarse mediante \textit{mutexs} y \textit{variables de condición}.  

\subsection{Variables de condición}
  Las \textit{variables de condición} son primitivas de sincronización utilizadas en asociación a un mutex con el fin de sincronizar dos o mas \textit{threads} en función de un predicado. La idea es que al realizar un \textit{wait} bloqueante sobre la \textit{variable de condición} (asociando cierto \textit{mutex} en dicho proceso), automaticamente (y atomicamente) se libere el mutex asociado, de manera que otro \textit{thread} a la espera de dicho \textit{mutex} continue su ejecución hasta cumplir cierta condición y luego el mismo despertiete nuestra \textit{variable de condición} por medio de alguna operación (por ejemplo un \textit{signal} o un \textit{broadcast}); vale aclarar que a diferencia de los \textit{semaforos} clásicos, en este caso el efecto de estas últimas operaciones no es acumulativo, sino instantaneo. Tras esto, el \textit{mutex} se vuelve a bloquear y el/los \textit{thread} que hayan estado esperando a la \textit{variable de condición} se despiertan y continuan su ejecución. Es importante aclarar que mas de un \textit{thread} podría estar esperando a cierta \textit{variable de condición} pero esperando que se cumplan distintas condiciones, por lo que podrían llegar a despertarse indebidamente; a estos casos se los llama \textit{Spurious wakeups}, y por ello es recomendable chequear la condición no solo antes sino después del \textit{wait} a la \textit{variable de condición}, tras lo cual volveremos a bloquearnos en caso de detectar que no se cumpla nuestra condición o predicado.

\subsection{Pseudo-código}

Para implementar este mecanismo, hemos diseñado la clase "RWLock", en ella tendremos los siguientes atributos:
\begin{itemize}
 \item \textit{readers}, un entero sin signo que registra la cantidad de lectores actuales (que hayan tomado el permiso de lectura y todavia no lo hayan liberado).
 \item \textit{writer}, un \textit{flag} que indica si alguien tomo el permiso de escritura.
 \item \textit{condition} y \textit{lock\_mutex}, son la \textit{variable de condición} y su \textit{mutex} asociado.
 \item \textit{lock\_writer}, un \textit{mutex} de seguridad, usado para preservar la estabilidad de la variable \textit{writer}. 
 \item \textit{lock\_reader}, un \textit{mutex} de seguridad, usado para preservar la estabilidad de la variable \textit{reader}. 
\end{itemize}

Tendremos además un metodo para cada \textit{lock} y \textit{unlock} de lectura y escritura con el siguiente \textit{pseudo-código}:

\begin{itemize}
\item \textbf{rlock()}:
	\begin{enumerate}
	\item Espero a tomar el \textit{lock\_mutex}.
	\item Mientras haya alguien escribiendo (es decir mientras este arriba el flag de 		\textit{writer}) me mantengo esperando a \textit{condition}.
	\item Ya no hay nadie escribiendo, asi que puedo leer, espero a tomar el \textit{lock\_reader}
	\item Me sumo a los lectores actuales incrementando la variable \textit{readers}.
	\item Libero el \textit{lock\_reader}.
	\item Libero el \textit{lock\_mutex}.
	\end{enumerate}
	
\item \textbf{wlock()}:
	\begin{enumerate}
	\item Espero a tomar el \textit{lock\_mutex}.
	\item Mientras haya alguien escribiendo (es decir mientras este arriba el flag de 		\textit{writer}) me mantengo esperando a \textit{condition}.
	\item Tomo el \textit{lock\_writer}.
	\item Indico que se esta escribiendo o se esta intentando escribir, levantando el \textit{flag} de \textit{writer}.
	\item Libero el \textit{lock\_writer}.
	\item Mientras haya alguien leyendo, es decir \textit{readers} sea mayor a 0, me mantengo esperando a \textit{condition}.
	\item Libero el \textit{lock\_mutex}.
	\end{enumerate}
	
\item \textbf{runock()}: 
	\begin{enumerate}
	\item Espero a tomar el \textit{lock\_mutex}.
	\item Espero a tomar el \textit{lock\_reader}
	\item Como ya no voy a leer decremento \textit{readers}.
	\item Libero el \textit{lock\_reader}.
	\item Si era el último lector (\textit{readers} vale cero), lo hago saber mediante un \textit{broadcast} sobre \textit{condition}.
	\item Libero el \textit{lock\_mutex}.
	\end{enumerate}
	
\item \textbf{wunlock()}:
	\begin{enumerate}

	\item Tomo el \textit{lock\_writer}.
	\item Indico que ya no hay nadie escribiendo, levantando el \textit{flag} de \textit{writer}.
	\item Libero el \textit{lock\_writer}.
	\item Hago saber que ya no hay nadie escribiendo despertando a los que esperen a \textit{condition}, mediante un \textit{broadcast}.
	\item Libero el \textit{lock\_mutex}.
	\end{enumerate}

\end{itemize}


\subsection{Implementacion}
\begin{lstlisting}[language=C++, breaklines=true]
class RWLock {
    public:
        RWLock();
        void rlock();
        void wlock();
        void runlock();
        void wunlock();

    private:
	    pthread_mutex_t lock_mutex;
        pthread_mutex_t lock_writer;
	    pthread_mutex_t lock_reader;
	    pthread_cond_t condition;	
	    bool writer;
	    unsigned int readers;

};

RWLock :: RWLock() {

	pthread_mutex_init(&(this->lock_mutex), NULL);
	pthread_mutex_init(&(this->lock_writer), NULL);
	pthread_mutex_init(&(this->lock_reader), NULL);
	pthread_cond_init (&(this->condition), NULL);
	writer = false;
	readers = 0;
}

void RWLock :: rlock() {

	pthread_mutex_lock(&(this->lock_mutex));

	while (writer) 
		pthread_cond_wait(&(this->condition), &(this->lock_mutex));
		
	pthread_mutex_lock(&(this->lock_reader));
	readers++;
	pthread_mutex_unlock(&(this->lock_reader));
	
	pthread_mutex_unlock(&(this->lock_mutex));
}

void RWLock :: wlock() {

	pthread_mutex_lock(&(this->lock_mutex));
	
	while (writer)
		pthread_cond_wait(&(this->condition), &(this->lock_mutex));

	pthread_mutex_lock(&(this->lock_writer));
	writer = true;
	pthread_mutex_unlock(&(this->lock_writer));

	while (readers > 0) 
		pthread_cond_wait(&(this->condition), &(this->lock_mutex));
	
	
	pthread_mutex_unlock(&(this->lock_mutex));
}

void RWLock :: runlock() {

	pthread_mutex_lock(&(this->lock_mutex));
	
	pthread_mutex_lock(&(this->lock_reader));
	readers--;
	pthread_mutex_unlock(&(this->lock_reader));
	
	if (readers == 0)
		pthread_cond_signal(&(this->condition));
	
	pthread_mutex_unlock(&(this->lock_mutex));
}

void RWLock :: wunlock() {

	pthread_mutex_lock(&(this->lock_writer));
	writer = false;
	pthread_mutex_unlock(&(this->lock_writer));
	
	pthread_cond_signal(&(this->condition));
}
\end{lstlisting}