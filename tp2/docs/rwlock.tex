\section{Read-Write Lock}

\subsection{Implementacion}
\begin{lstlisting}[language=C++, breaklines=true]
class RWLock {
    public:
        RWLock();
        void rlock();
        void wlock();
        void runlock();
        void wunlock();

    private:
	    pthread_mutex_t lock_mutex;
	    pthread_cond_t condition;	
	    bool writer;
	    unsigned int readers;

};

RWLock :: RWLock() {

	pthread_mutex_init(&(this->lock_mutex), NULL);
	pthread_cond_init (&(this->condition), NULL);
	writer = false;
	readers = 0;
}

void RWLock :: rlock() {

	pthread_mutex_lock(&(this->lock_mutex));

	while (writer) 
		pthread_cond_wait(&(this->condition), &(this->lock_mutex));
	readers++;
	
	pthread_mutex_unlock(&(this->lock_mutex));
}

void RWLock :: wlock() {

	pthread_mutex_lock(&(this->lock_mutex));
	
	while (writer)
		pthread_cond_wait(&(this->condition), &(this->lock_mutex));

	writer = true;

	while (readers > 0) 
		pthread_cond_wait(&(this->condition), &(this->lock_mutex));
	
	
	pthread_mutex_unlock(&(this->lock_mutex));
}

void RWLock :: runlock() {

	pthread_mutex_lock(&(this->lock_mutex));
	
	readers--;
	if (readers == 0)
		pthread_cond_signal(&(this->condition));
	
	pthread_mutex_unlock(&(this->lock_mutex));
}

void RWLock :: wunlock() {

	writer = false;
	pthread_cond_signal(&(this->condition));
}
\end{lstlisting}